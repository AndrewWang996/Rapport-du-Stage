\newpage
\section{Context: Hydrocarbon Exploration}

The processes of Reverse Time Migration and solving forward problems in creating images of the underground is primarily used in the field of hydrocarbon exploration. This relatively recent field has its origins in the 1900's, but most of the major oil pockets have already been discovered and now the job of hydrocarbon exploration becomes more difficult and necessary in underground zones that are smaller and harder to image. Add to this the difficulties of the often mountainous or forest-heavy terrain above the surface, and we can see that the hydrocarbon exploration is an increasingly complicated field in both the geophysical and mathematical domains. 

The cost of discovering a barrel has skyrocketed more than three-fold over the last decade due to these increasingly inhospitable environmental conditions and a fairly constant reduction of existing reserves at $5-15\%$ per year \cite{hydrocarbonExplorationCosts}. In addition, political risks associated with the discovery and drilling of new oil wells have been escalating. For example, the USA, Russia, Canada, Denmark, Iceland, and Norway have competing claims over the Arctic Circle where approximately a fifth of the world's recoverable oil is contained. Although Uganda has more than 2 billion barrels of oil, necessary negotiations between the government and oil companies have been slow. In 2009, 3D seismic surveys cost on average between \$40,000 and \$100,000 per square mile.

\subsection{Seismic Acquisition}



\subsection{Method of Exploration: Seismic Reflection}

\subsection{Processing the data}

\subsection{Seismic Waves}

\subsubsection{Waves in 3D}



















