
\newpage
\section{Research within INRIA}

\subsection{Five Fields of Research}

Within INRIA, there are five essential fields of research. Within each field, there are smaller subfields with project-teams dedicated to topics within those subfields. Listed here are all five fields of research and the subfields that are listed at the Bordeaux Sud-Ouest Center of Research.
\begin{enumerate}

\item \href{http://www.inria.fr/en/research/research-fields/five-fields-of-research/applied-mathematics-computation-and-simulation}{Applied Mathematics, Computation and Simulation}

\subitem Numerical schemes and simulations
\subitem Stochastic approaches
\subitem Optimization, machine learning and statistical methods

\item \href{http://www.inria.fr/en/research/research-fields/five-fields-of-research/algorithmics-programming-software-and-architecture}{Algorithmics, Programming, Software and Architecture}

\subitem Algorithmics, Computer Algebra and Cryptology
\subitem Embedded and Real-time Systems

\item \href{http://www.inria.fr/en/research/research-fields/five-fields-of-research/networks-systems-and-services-distributed-computing}{Networks, systems and services, distributed computing}

\subitem Distributed and High Performance Computing
\subitem Distributed programming and Software Engineering

\item \href{http://www.inria.fr/en/research/research-fields/five-fields-of-research/perception-cognition-interaction}{Perception, Cognition \& Interaction}

\subitem Interaction and visualization
\subitem Robotics and Smart environments

\item \href{http://www.inria.fr/en/research/research-fields/five-fields-of-research/digital-health-biology-and-earth}{Digital Health, Biology and Earth}

\subitem Earth, Environmental and Energy Sciences
\subitem Modeling and Control for Life Sciences
\subitem Computational Biology
\subitem Computational Neuroscience and Medicine

\end{enumerate}


\subsection{Centers of Research}

\begin{enumerate}
\item \href{http://www.inria.fr/centre/bordeaux}{Bordeaux - Sud-Ouest}
\item \href{http://www.inria.fr/centre/grenoble}{Grenoble - Rhone-Alpes}
\item \href{http://www.inria.fr/centre/lille}{Lille - Nord Europe}
\item \href{http://www.inria.fr/centre/nancy}{Nancy - Grand Est}
\item \href{http://www.inria.fr/centre/paris-rocquencourt}{Paris - Rocquencourt}
\item \href{http://www.inria.fr/centre/rennes}{Rennes - Bretagne Atlantique}
\item \href{http://www.inria.fr/centre/saclay}{Saclay - Ile-de-France}
\item \href{http://www.inria.fr/centre/sophia}{Sophia Antipolis - Mediterranee}
\end{enumerate}

\subsection{Project Teams}
CP'ed from \href{http://www.inria.fr/en/research/research-teams/project-team-model}{Inria Project-team Model}

Ever since it first began, Inria has made use of an original research model founded on a basic entity:the project-team. This model is the component that structures the institute's research activities. Made up of around 20 individuals, the project-team is formed around a "scientific leader", who defines scientific objectives on a topic approved by the institute.


\subsection{Magique 3D}

C/P'ed from: \href{https://team.inria.fr/magique3d/research/}{magique3D/research}

\subsubsection{Purpose and Research Themes}
    Magique-3D is a joint project-team between Inria and the Department of Applied Mathematics (LMA) of the University of Pau in partnership with CNRS. The mission of Magique-3D is to develop and validate efficient solution methodologies for solving complex three-dimensional geophysical problems, with a particular emphasis on problems arising in seismic imaging, in response to the local industrial and community needs. Indeed, as it is well known, the region of Pau has long-standing tradition in the Geosciences activities. However, in spite of the recent significant advances in algorithmic considerations as well as in computing platforms, the solution of most real-world problems in this field remains intractable. Hence, there is a scientific need of pressing importance to design new numerical methods for solving efficiently and accurately wave propagation problems defined in strongly heterogeneous domains.


The research record of Magique-3D group covers a large spectrum of accomplishments in the field of wave propagation including (a) the design, validation, and performance assessment of a class of DG-methods for solving efficiently high frequency wave problems, (b) the construction, convergence analysis, and performance assessment of various absorbing-type boundary conditions that are key ingredients for solving problems in infinite domains, and (c) the development of asymptotic models that are the primary candidate in the presence of heterogeneities that are small compared to the wave length.

    Magique-3D has built strong collaborations and partnerships with various institutions including (a) local industry (TOTAL), (b) national research centers (ONERA and CEA), and (c) international academic partnerships (e.g. Interdisciplinary Research Institute for the Sciences (IRIS) at California State University, Northridge, USA; University of Pays Basque at Bilbao, Spain; University of Novosibirsk, Russia).

\subsubsection{Mathematical modeling of wave propagation}

    The development of migration software with preserved amplitudes is of great interest for imaging geological structures based on the propagation of seismic waves. Most of the geophysicists use the Kirchhoff formalism with a posteriori corrections of the amplitude. Magique-3D proposes instead to evaluate the exact amplitude directly by developing more complete modeling techniques. This implies the construction of new models and the analysis of their qualitative and numerical properties. New models must incorporate absorbing conditions in order to be able to compute the solution in bounded domains. The construction of such conditions is optimized in order to improve the accuracy of the numerical solution and/or reduce the computational cost.
    

\subsubsection{Numerical simulation, parallel computing, GRID computing}

    The spectral element method (SEM) has recently shown its efficiency for the computation of synthetic seismograms compared to more classical approaches such as finite difference schemes. Magique-3D uses the SEM to quantify the effects of both topography and variations of geological structures on the propagation of seismic waves. Magique-3D also considers simplified inverse problems for 3D structures, which makes it possible for instance to analyze the propagation of surface waves in weathered zones in the context of active seismic experiments performed by the petroleum industry. Magique-3D also intends to develop a finite element method optimized to run on a parallel computer for the study of geomorphology. This project could in part be done jointly with Scalapplix. Magique-3D also intends to study the propagation of elastic waves in fractured media by coupling quasi-analytic methods near the fractures with a finite element method in the surrounding medium. The numerical methods involved in this work all result in a high computational cost, and we therefore want to benefit from recent technological advances by developing algorithms that can not only run on very large parallel computers but also on so-called "grids" of computers ("GRID computing").

