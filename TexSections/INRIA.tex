\newpage
\section{Inria}

The Institut National de Recherche en Informatique et Automatique is a public establishment dedicated to technological research and software development. INRIA's goal is to create a network of talent both french and international concerning its five fields of research: http://www.inria.fr/en/research/research-fields/five-fields-of-research. 

\subsection{History}

C/P'ed from: \href{http://www.inria.fr/en/institute/inria-in-brief/history-of-inria}{History of INRIA}

The creation of IRIA, the precursor of INRIA, along with a number of other organisations, was a symbol of the proactive policies to develop cutting-edge technologies and the means to produce them. This prompted the country to create a national champion in the form of the CII (the Compagnie Internationale pour l'Informatique) in December 1966, following the takeover of Bull by General Electric. The creation of IRIA was also a response to a desire to develop an institute that was close to industry and capable of educating the country in the fields of computer science and control.


1967 was the year of the Plan Calcul, a French governmental programme aimed at promoting a national computer manufacturing industry and associated research activities. This programme included the creation of a new government body responsible for computer science, the Délégation à l'informatique, and a new research institute, IRIA, under the leadership of Michel Laudet. IRIA was a new kind of organisation, close to the private sector, which prefigured a new relationship between the public sector and industry. It was designed to be the active arm of the CII.


As soon as it was created, IRIA began to organise international conferences, inviting big names from the fields of computer science and applied mathematics. The institute rapidly earned itself an international reputation. One of its priorities was training: summer schools were set up with EDF and the CEA, as well as a training centre: the centre for practical studies in computer science and control (CEPIA), which delivered 5,000 hours of classes in its first year alone.


In 1973, André Danzin organised IRIA around SESORI (the department for the consolidation and orientation of computer science research, led by Michel Monpetit, which was responsible for links with the Plan Calcul) and Laboria (the computer science and control research laboratory) with Jacques-Louis Lions. Laboria was organised around research projects, with its own resources, objectives, project managers and schedules. But with only 80 researchers, Laboria was nowhere near big enough.


The period from 1974 to 1979 saw both a maturing of the institute and a few missed opportunities. Under the leadership of Jacques-Louis Lions, Laboria built itself a strong identity. Under the presidency of Valéry Giscard d'Estaing, the European collaborations came under threat, while a lack of flexibility and resources prevented the institute from making quick progress. There was real dissemination of knowledge, the institute developed a reputation in other countries, long-term strategies were established, and IRIA began to push towards Rennes and think about Sophia Antipolis. But in the late seventies it was still struggling to find its place.


SESORI launched a number of pilot projects to produce products that could be used in industry. One example was its computer-aided design and drawing mission (MICADO), which made it possible to coordinate research in CAD. SESORI was responsible for coordinating national work on various themes ranging from robotics to breakdown prevention, from shape recognition to digital image processing.



At Laboria, the Cyclades project explored innovative solutions for creating a computer network based on the Cigale packet switching data network. Presentations in France, Europe and the United States put IRIA at the forefront of the world scene in this field. Despite this, the project was suspended in 1976. The Spartacus project, a collaboration with Inserm (the French National Institute for Health and Medical Research), the CNRS (the French National Centre for Scientific Research) and the CEA (the French Atomic Energy Commission), aimed to develop a system allowing tetraplegics to regain some independence.

In 1979, decentralisation threatened IRIA's existence: there was talk of relocation to Sophia Antipolis or a merger with IRISA in Rennes (which had been created with IRIA's help in 1975). Eventually, Jacques-Louis Lions was able to keep the institute at Rocquencourt, and an 'N' was added to its name. In accordance with the Decree of 27 December 1979, the institute would henceforth be known as Inria.


The appointment of Jacques-Louis Lions as Inria Chairman in 1980 marked a turning point. In the 1980s, the institute, still with very limited resources, developed a model centred around the excellence of its research, with the constant aim of ensuring technology transfer to industry (creation of innovative businesses in strategic sectors). Inria created a network for European researchers (ERCIM, European Research Consortium for Informatics and Mathematics ) and played a major role in the development of the Internet in Europe. After a lot of trial and error, Inria now had clear aims and a strong international reputation.


As part of a process of decentralisation and regional development, the National Institute for Research in Computer Science and Control built centres all over the country:

Irisa and then the Rennes research unit from 1975;
the Sophia-Antipolis research unit in 1983;
the Lorraine research unit / Loria in 1986;
the Rhône-Alpes research unit in 1992;
the Futurs research unit from 2003, incubating 3 future units in Bordeaux, Lille and Saclay.
20 years of business creation

Inria's innovative technology transfer activities included the filing of patents, agreements with industrial partners, the running of Consortia, and support for innovative businesses. Between 1984 and 2004, 80 businesses were created, 45 of which still exist. The number of international collaborations grew rapidly.


4-year strategic plans, corresponding to 4-year contracts with the State, committed Inria to a number of priorities and performance objectives in terms of research, innovation and technology transfer on a global scale.
The first strategic plan (1994-1998) focussed on four research themes: networks and systems, software engineering and symbolic computing, human-machine interaction, images, data and knowledge, and simulation and optimisation of complex systems.


Futurs batiments INRIA Bordeaux Sud-Ouest
After celebrating its fortieth anniversary in 2007, Inria continues to expand. Between 1999 and 2009, its workforce doubled . Three new research centres in Saclay, Bordeaux and Lille were added to the 5 existing centres in Rocquencourt,  Rennes,  Sophia Antipolis, Nancy and  Grenoble. Firmly rooted in local industrial and academic ecosystems, Inria is pursuing an increasing involvement in the European Research Area. With a resolutely international outlook, it is contributing to the global profile of computational sciences. Convinced that the future of our societies lies in digital technology, Inria is tackling research subjects crucial to the social issues of today.

Inria is helping to bolster the competitiveness of the economy in a sector that creates a large number of jobs. Its policy of partnerships with industry and SMEs illustrates its proactive approach to technology transfer.

In order to disseminate scientific information and knowledge more effectively, Inria is committed to open access and sharing of data. The Institute is also involved in the field of freeware.

Furthermore, it is one of the founder members of the Alliance des sciences et technologies du numérique (ALLISTENE), the aim of which is to eliminate barriers between research players and develop partnership initiatives.


More than half of Inria project-teams are involved in the European Framework Programmes for Research and Development (FPRD). The European Commission ranks Inria in the top ten contributing organisations. As part of FPRD 7, Inria has helped to identify two major scientific challenges: the Internet of the future and the digital patient. In 2006, Inria signed up to the European Charter for Researchers.


The strategic plan for 2008-2012 sets out the scientific objectives for the immediate future, based on the challenges of the 21st century. Inria is strengthening and diversifying its partnerships with other scientific disciplines and the business world (strategic partnerships), in France and in Europe, in the USA and with emerging countries (China, India, South America, Africa).

\subsection{Organization}

Click here: \href{http://www.inria.fr/en/institute/organisation}{Organization of Inria}

\begin{figure}[ht]
	\centering
	\includegraphics[width=0.8\textwidth]{Images/INRIA.jpg}
	\caption{Inria Logo\protect\footnotemark}
\end{figure} 

\footnotetext{This image belongs to Inria, not me.}



